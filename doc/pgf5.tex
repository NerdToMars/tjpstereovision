% Latex pgf picture
% Shows graph of pixels {fa,p,q,r,s,t,f0}
% With cut C, where:
% C = {p-fa,q-fa,r-fa,r-s,s-f0,s-t,t-fa}
\begin{pgfpicture}
\pgfsetlinewidth{1pt}
\pgfnodebox{p}[stroke]{\pgfxy(1,2)}{\textcolor{black}{P}}{4pt}{4pt}
\pgfnodebox{q}[stroke]{\pgfxy(3,2)}{\textcolor{black}{Q}}{4pt}{3pt}
\pgfnodebox{r}[stroke]{\pgfxy(5,2)}{\textcolor{black}{R}}{4pt}{4pt}
\pgfnodebox{s}[stroke]{\pgfxy(7,2)}{\textcolor{black}{S}}{5pt}{4pt}
\pgfnodebox{t}[stroke]{\pgfxy(9,2)}{\textcolor{black}{T}}{4pt}{4pt}

\pgfsetlinewidth{.5pt}
\color{black}
\pgfnodebox{f0}[stroke]{\pgfxy(5,0)}{$f^0$}{4pt}{4pt}
\pgfnodebox{a}[stroke]{\pgfxy(5,4)}{{$f^\alpha$}}{4pt}{4pt}

\pgfnodeconnline{p}{q}
\color{red}
\pgfnodeconnline{q}{r}
\color{black}
\pgfnodeconnline{r}{s}
\pgfnodeconnline{s}{t}


\pgfnodeconnline{f0}{p}
\pgfnodeconnline{f0}{q}
\color{red}
\pgfnodeconnline{f0}{r}
\pgfnodeconnline{f0}{s}
\pgfnodeconnline{f0}{t}

\color{red}
\pgfnodeconnline{a}{p}
\pgfnodeconnline{a}{q}
\color{black}
\pgfnodeconnline{a}{r}
\pgfnodeconnline{a}{s}
\pgfnodeconnline{a}{t}

\color{red}
\pgfsetdash{{0.2cm}{0.2cm}{0.2cm}{0.2cm}}{0cm}
\pgfsetlinewidth{1pt}

\pgfmoveto{\pgfxy(1,3)}
\pgflineto{\pgfxy(3.1,3)}

\pgfcurveto{\pgfxy(4.3,3)}{\pgfxy(4.4,2)}{\pgfxy(4.4,2)}
\pgfcurveto{\pgfxy(4.4,1.3)}{\pgfxy(5,1.3)}{\pgfxy(5.5,1.3)}
\pgflineto{\pgfxy(10,1.3)}

\pgfstroke
\pgfputat{\pgfxy(1.5,3.2)}{\pgfbox[center,center]{$\mathcal{C}$}}
\end{pgfpicture}