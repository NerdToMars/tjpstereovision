\documentclass{beamer}

%\colorlet{structure}{red!65!black}
%
%\beamertemplateshadingbackground{blue!40}{white}
%
\usepackage{beamerthemesplit}
%\usepackage{hyperref}
\usetheme{Berkeley}
\useinnertheme{circles}
\usecolortheme{seagull}

\title[Stereo vision]{Stereo Vision using the OpenCV library}
\author[Dr\"oppelmann \and Hueting \and Latour \and \\Van der Veen]{Sebastian Dr\"oppelmann \\ Moos Hueting \\ Sander Latour \\ Martijn van der Veen}
\institute{University of Amsterdam}
\date{June 2010}
\subject{Computer vision}

\begin{document}

\frame
{
  \titlepage
}

\section{Preface}

\frame
{
 \frametitle{Goal}
 \begin{block}{Goal}
   Generating a disparity depth map of the environment using stereo vision.
 \end{block}
}

\frame{
 \frametitle{Why it is interesting}
   A depthmap can be used for various purposes:
   \begin{itemize}
    \item 3D modeling of 2D images
    \item Tracking of objects
    \item Recognising front objects
    \item As information about the environment in path planning
   \end{itemize}
}


\section{Problems}

\frame{
 \frametitle{Theoretical problems}
   Stereo vision in a real life environment can be split up in several subproblems:
   \begin{itemize}
    \item Camera calibration problems
    \item Generating epipolar line
    \item Matching points in both images
    \item Occlusion
   \end{itemize}
}

\section{Approach}

\frame
{
  \frametitle{Approach}
  \begin{itemize}
    \item Camera calibration
    \item Epipolar geometry
     \item Dense stereo algorithms
      \begin{itemize}
       \item Graph Cut
       \item Belief Propagation
       \item Region Based
      \end{itemize}
    \item Using the OpenCV library
  \end{itemize}
}

\frame{
  \frametitle{Separate goals}
  The goal can be seperated into two independent subgoals:

  \begin{itemize}
   \item \textbf{Calibration and rectification} Starting with two cameras and building a rectified image
   \item \textbf{Dense stereo} Starting with a rectified image and building a dense disparity map \\ \textit{Can use an external dataset}
  \end{itemize}
}

\frame
{
  \frametitle{Practical problems}
  \begin{itemize}
   \item Getting webcams to work
   \item Learning OpenCV
   \item Selecting and understanding the right dense stereo algorithm
  \end{itemize}
}

\section{Planning}

\frame
{
  \frametitle{Tasks}
  \begin{itemize}
    \item Martijn and Moos
    \begin{itemize}
      \item Camera calibration
      \item Epipolar geometry
    \end{itemize}
    \item Sander and Sebastian
    \begin{itemize}
      \item Finding corresponding points
      \item Generating depth map
    \end{itemize}
  \end{itemize}
}

\frame
{
  \frametitle{Planning}
  \begin{itemize}
    \item Week 1
      \begin{itemize}
        \item Reading literature
        \item Getting webcams to work
        \item Choosing dense algorithm
      \end{itemize}
    \item Week 2 and 3
      \begin{itemize}
        \item Implementing
          \begin{itemize}
            \item Camera calibration
            \item Rectification of images using epipolar geometry
            \item Dense disparity map algorithm
          \end{itemize}
        \item Halfway report
      \end{itemize}
    \item Week 4
      \begin{itemize}
        \item Optimizing and testing
        \item If there's enough time left
          \begin{itemize}
            \item Generate 3D image of environment
            \item Remove background using dense disparity map
          \end{itemize}
      \end{itemize}
  \end{itemize}
}
\end{document}
