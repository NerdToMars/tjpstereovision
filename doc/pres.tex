\documentclass{beamer}

%\colorlet{structure}{red!65!black}
%
%\beamertemplateshadingbackground{blue!40}{white}
%
\usepackage{beamerthemesplit}
%\usepackage{hyperref}
\usetheme{Berkeley}
\useinnertheme{circles}
\usecolortheme{seagull}

\title[Stereo vision]{Stereo Vision using the OpenCV library}
\author[Dr\"oppelmann \and Hueting \and Latour \and \\Van der Veen]{Sebastian Dr\"oppelmann \\ Moos Hueting \\ Sander Latour \\ Martijn van der Veen}
\institute{University of Amsterdam}
\date{June 2010}
\subject{Computer vision}

\begin{document}

\frame
{
  \titlepage
}

\section{Preface}

\frame
{
 \frametitle{Goal}
 \begin{block}{Goal}
   Generating a depth map of the environment using stereo vision
 \end{block}
}

\frame
{
  \frametitle{The Why}
  \begin{itemize}
    \item Normally: $\mathbb{R}^3 \rightarrow \mathbb{R}^2$
    \item Now: $\mathbb{R}^3 \rightarrow \mathbb{R}^3$
  \end{itemize}
}

\section{Approach}

\frame
{
  \frametitle{Approach}
  \begin{itemize}
    \item Camera calibration
    \item Epipolar geometry
    \item TOEDOE
  \end{itemize}
}

\frame
{
  \frametitle{Difficulties}
  \begin{itemize}
    \item Practical problems
    \begin{itemize}
      \item Getting webcams to work
      \item Learning OpenCV
    \end{itemize}
    \item Choosing algorithms for depthmap generating
  \end{itemize}
}

\section{Planning}

\frame
{
  \frametitle{Tasks}
  \begin{itemize}
    \item Martijn and Moos
    \begin{itemize}
      \item Camera calibration
      \item Epipolar geometry
    \end{itemize}
    \item Sander and Sebastian
    \begin{itemize}
      \item Finding corresponding points
      \item Generating depth map
    \end{itemize}
  \end{itemize}
}

\frame
{
  \frametitle{Planning}
  \begin{itemize}
    \item Week 1
      \begin{itemize}
        \item Reading literature
        \item Getting webcams to work
        \item TOEDOE
      \end{itemize}
    \item Week 2
      \begin{itemize}
        \item Proggen
      \end{itemize}
  \end{itemize}
}

\end{document}
