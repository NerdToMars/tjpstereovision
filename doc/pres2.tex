\documentclass{beamer}

%\colorlet{structure}{red!65!black}
%
%\beamertemplateshadingbackground{blue!40}{white}
%
\usepackage{beamerthemesplit}
%\usepackage{hyperref}
\usetheme{Berkeley}
\useinnertheme{circles}
\usecolortheme{seagull}

\title[Stereo vision]{Stereo Vision using the OpenCV library}
\author[Dr\"oppelmann \and Hueting \and Latour \and \\Van der Veen]{Sebastian Dr\"oppelmann \\ Moos Hueting \\ Sander Latour \\ Martijn van der Veen}
\institute{University of Amsterdam}
\date{June 2010}
\subject{Computer vision}

\begin{document}

\frame
{
  \titlepage
}

\section{Preface}

\frame
{
 \frametitle{Goal}
 \begin{block}{Goal}
   Generating a disparity depth map of the environment using stereo vision.
 \end{block}
}

\frame{
 \frametitle{Why it is interesting}
   A depthmap can be used for various purposes:
   \begin{itemize}
    \item 3D modeling of 2D images
    \item Tracking of objects
    \item Recognising front objects
    \item As information about the environment in path planning
   \end{itemize}
}

\section{Depth Map Algorithms}


\frame
{
  \frametitle{Depth Map Algorithms}
  
  Tested with standard  OpenCV algorithms\\
  Used datasets from Middlebury\\

  \begin{itemize}
    \item Graph Cut
    \item Block Matching
    \item Semi Global Block Matching
  \end{itemize}
  
}

\frame {
  \frametitle{New Problems}
  \begin{itemize}
    \item Parameters of Algorithms
    \item understand and choose right occlusion handling
    \item implementation very bad with twisted cameras
  \end{itemize}
}

\frame {
  \frametitle{Graph Cut}
      \begin{itemize}
      \item slow
      \item not very precise
      \end{itemize}
      \importimage{disp_tsukuba_orig}
      \importimage{gc_tsukuba_own}
}


\frame {
  \frametitle{Block Matching}
    \begin{itemize}
      \item faster
      \item interpretation missing
    \end{itemize}
    \importimage{disp_tsukuba_orig}
    \importimage{bm_tsukuba_own}
}

\frame{
    \frametitle{Semi Global Block Matching}
    \begin{itemize}
      \item not in Python (used C++ implementatie)
      \item fastest
      \item good depthmap
      \item much noise (needs tuning)
    \end{itemize}
    \importimage{disp_tsukuba_orig}
    \importimage{sgbm_tsukuba_own}
}



\section{Planning}
\frame
{
  \frametitle{Planning}
  \begin{itemize}
    \item Week 1
      \begin{itemize}
        \item {\bf done} Reading literature
        \item {\bf done} Getting webcams to work
        \item {\bf done} Choosing dense algorithm
      \end{itemize}
    \item Week 2 and 3
      \begin{itemize}
        \item Implementing
          \begin{itemize}
            \item \emph{done} Camera calibration
            \item {\bf done} Rectification of images using epipolar geometry
            \item \emph{done} Dense disparity map algorithm
          \end{itemize}
        \item Halfway report
      \end{itemize}
    \item Week 4
      \begin{itemize}
        \item Optimizing and testing
        \item If there's enough time left
          \begin{itemize}
            \item Generate 3D image of environment
            \item Remove background using dense disparity map
          \end{itemize}
      \end{itemize}
  \end{itemize}
}

\frame {
  \frametitle{new planning - week 3}
  \begin{itemize}
    \item Martijn en Moos
    \begin{itemize}
      \item Fine tuning calibration
      \item cropping of rectified images
    \end{itemize}
    \item Sebastian en Sander
    \begin{itemize}
      \item Fine tuning parameters
      \item Completely understand the algorithms
      \item depth map normalize
    \end{itemize}
}

\end{document}
